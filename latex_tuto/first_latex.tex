\documentclass[twocolumn]{book}

\usepackage[latin1]{inputenc}    
\usepackage[T1]{fontenc}
\usepackage[francais]{babel}   
\usepackage{layout}  
      
\begin{document}

\layout
\section{Un fort beau chapitre}
cinq ou six gros paragraphes de faux texte.

\paragraph TeX is a computer program for typesetting documents, created by D. E. Knuth. It takes a suitably prepared computer file and converts it to a form that may be printed on many kinds of printers, including dot-matrix printers, laser printers and high-resolution typesetting machines. A number of well-established publishers now use TeX in order to typeset books and mathematical journals.

Simple documents that do not contain mathematical formulae or tables may be produced very easily: the body of the text is typed in essentially unaltered (though observing certain rules regarding quotation marks and punctuation dashes). Typesetting mathematics is somewhat more involved, but even here TeX is comparatively straightforward to use when one considers the complexity of some of the formulae that it is required to typeset.

LaTeX, written by L. B. Lamport, is one of a number of `dialects' of TeX. It is particularly suited to the production of long articles and books, since it has facilities for the automatic numbering of chapters, sections, theorems, equations etc., and also has facilities for cross-referencing. It is probably one of the most suitable version of LaTeX for beginners to use.

\paragraph This introduction describes basic features of LaTeX2e, released in 1994. Further information on LaTeX is to be found in the 2nd edition of LaTeX User's Guide and Reference Manual by Leslie Lamport, and in The LaTeX Companion by Michel Goossens, Frank Mittelbach and Alexander Samarin.
\end{document}