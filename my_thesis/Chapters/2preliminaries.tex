% Chapter 1

\chapter{Preliminaries} % Main chapter title

\label{Chapter2} % For referencing the chapter elsewhere, use \ref{Chapter1} 

\lhead{Chapter 2. \emph{Preliminaries}} % This is for the header on each page - perhaps a shortened title

%----------------------------------------------------------------------------------------

\section{Definitions and notations}

In this work, we take an interest on datasets containing smartphone logs of a user. We refer to those datasets as $data$, $smartphone$ $logs$, $user$ $logs$ or $dataset$.

The particularity of these logs is that they contain different information sources and types. They contain for example information about the location of the user, the activities he is doing, the applications he uses, the settings he puts in the phone, the notifications he receives, the bluetooth devices that connects to and the external devices that he connects to the phone (headset, powerplug).
We refer to these different types as $features$ or $types$.

Smartphone logs contains multiple features ($Location$, $Notification$) where each feature can take multiple values. For example, feature $Location$ can take values $Home$, $Work$, the feature $Activity$ can take values as $on\_foot$, $on\_bicycle$, $in\_vehicle$ and $still$. We refer to the values of a feature $fname$ as the $values$ of feature $fname$ or the $vocabulary$ of feature $fname$ or even the $dictionary$ of feature $fname$.

We note the set $F= \left \{ 1,.., J \right \}$ as the set representing all the J different features. present in user logs. Here, we refer to features as ids and no longer as names. We note $f \in F$ as the feature number $f$.

Similarly, we note the set $V_f = \left \{ 1,.., I_f \right \} $ as the set representing the $I_f$ different values that can be taken by $f, f \in F$. We refer to values as ids and no longer as names. Note that $I_f$ represents the vocabulary size of feature $f$. $v \in V_f$ indicates the value number $v$ of the feature $f$.

Thus, the complete values space of the dataset is completely defined by the sets $F$ and $V_1,...,V_f$. We refer to this $complete$ $values$ $space$ as the $language$ defined by the dataset. Note that the size of the language is just the sum of the vocabulary sizes of all the features ($\sum_{f=1}^{F} I_f$).

A $data point$, also called a $realization$ is then represented as the pair $(f,v)$ which means the $v^{th}$ value of the $f^{th}$ feature.

The user logs is nothing than a set of multiple realizations where each data point $(f,v)$ is linked with a time stamp indicating its time of occurence.

Let the $record$ be the set of realizations that occured during a certain time frame. Thus a record containing n realizations $\left \{ (y_1,w_1),..., (y_N,w_N) \right \}$ that all occured in the same time frame can be represented as vector $\mathbf{r} = [(y_1,w_1),..., (y_N,w_N)]$. Here $y_{n} \in F$ denotes the feature of the $n^{th}$ realization and $w_n$ the value of the $n^{th}$ realization (i.e the value taken by the feature $y_n$).

Using the record definition, smartphone logs can be represented as a set of records where each record contains the data points that occured during a certain period in the time. A period could be for example 1 hour. In this case, each record represents 1 hour of the period of observation of a user.

Using this representation a smartphone logs can be represented as a set $R = \left \{ \mathbf{r_1},...,\mathbf{r_M} \right \} $, of $M$ records, where each record $\mathbf{r_m}$ has a size $N_m$, for $m \in \left \{ (1,...,M \right \}$.

We call $R$ a $corpus$ and when considering this representation, we may refer to smartphone logs as a $corpus$.

%----------------------------------------------------------------------------------------
\section{Problem Statement}
The goal that drives our work is to extract the behaviors and habits of a user from his smartphone logs data.
In fact, all individuals have their own behaviors and habbits that repeat periodically over time. For example, an individual generally works during the week days, to sleep at home during the night. He might also be playing sport once a week, listening to music while driving or reading news when he is in the bus.

The question is, without previously knowing any behavior of the user, how well can we discover and extract the behaviors by analyzing a user logs.
We note that each of the examples of behaviors given above can be represented by the smatphone logs as a set containing a specific types of realizations. For example, the behavior working on the week days can be represented by the set $\{$ $(day: monday),$ $(day: tuesday),$$...,$$(day: friday),$$(hour: 8am-6pm),$ $(location: work) \}$.  The behavior listening to music while driving could be represented by the set $\{ (Activity: in\_vehicle),$ $(Application\_ Launch: music)\}$.

More generally, most of humans' behaviors (that can be expressed by smartphone logs) that one could think of can be represented as a group containing some realizations. For this reason and starting from this observation, the approach that we decide to take in our work is to consider a behavior as a set of data logs realizations.

We define a behavior z as a set of data realizations.  By considering smartphone logs as a data showing a subsample of a user s life, the task of finding the behaviors that are the most representative of his life is equivalent to find the groups of realizations that are the most descriptive of the smartphone data. The terms $habbits$, $class$ or $cluster$ are also used to refer to a behavior.

Formulating our problem differently, we aim to find a set of $K$ classes $\{ z_1,...,z_K \} $, where each class is a group of realizations. This means that those classes represents short descriptions of the dataset while preserving the essential statistical relationships of the data. It is this requirement that imposes the classes to represent real user s behaviors.

In section 8, we discuss in more details what does this requirement mean, what does it imply and how it can be measured.
%----------------------------------------------------------------------------------------


%----------------------------------------------------------------------------------------

\section{Related work}
In this section we expose the different researches that have been done using smartphone logs and that are relevant to our work. Smartphone logs are very rich datasets containing multiple types of informations and with an abundant quantity, thus they can be used to achieve different goals and to tackle a large range of different problems. Etter and al. \cite{wheretogo} used smartphone logs to predict the next location to which the user will go. Zhu and al. \cite{mobapp} used them to classify smartphone applications. We detail below two works that are specifically relevant to ours.

Cao and al. \cite{interaction} addressed in 2010 the following problem: having a dataset of smartphone logs, they tried to discover the context that causes a user to have a certain interaction with the phone. An interaction could be "$listening$ $to$ $music$", $"reading$ $news"$, $"having$ $a$ $message$ $session"$. A context is a set composed by the features that are not phone interactions. For example a Context is $C= \{$$"Is$ $holiday$$=$$Yes"$, $"Day period$$=$$morning"$, $"Phone$ $profile$$=$$silent"$, $"speed$$=$$High"\}$. They were for example interested in learning that the context $C$ usually implies the interaction $"reading$ $news"$.
\\To that end, they used association rules to learn the contextual information that leads to a phone interaction. The idea is to look through all the contextual features, build contexts and count how many times a context appears with a certain type of phone interaction. As checking all the combinations of the contextual features exponentially explode, thresholds $min\_support$ and $min\_confidence$ are used to only go through the promising contexts. \par

We can note that the problem tackled by \cite{interaction} is similar to our problem in the sense that they tried to learn a user behavior by processing his smartphone logs. However, they were interested in a specific type of behavior, the behavior of a user that leads him to make an interaction with his phone. In our work, we are considering smartphone logs as a subsample of a user's life and are interested in discovering the behaviors that drive his life (and not only the ones related to the interaction with his smartphone). Thus, while the two problems may seem similar at a first look, they are in reality very different. To be able to learn any kind of behavior, we make the choice to proceed by a fully unsupervised strategy. For example, we do not input any prior knowledge or general behavior that humans might have. For example we do not take profit from the fact that most of people have different behaviors during the week days and the week ends (i.e separate the week end from the week day). Indeed, if Bob is a farmer, he may prefer to rest of Fridays and not during the week ends.\par

In extension to this work, Ma et al. \cite{superbehaviors} tried to detect similarities between users based on their habits (2012). Based on the same dataset than \cite{interaction} and taking the results of \cite{interaction} as input (i.e having learned context interaction relations for each user individually), they tried to cluster the users by behavior similarities. This work could be used for context-aware recommendation systems. Context aware recommendation systems take into account the context of the user to give him the right recommendation at the right time.
\\To extract super- behaviors (which are clusters of behaviors) of users, they use $Linearly$ $Constrained$ $Bayesian$ $Matrix$ $Factorization$ ($LCBMF$) model described in \cite{superbehaviors}. This method is a bayesian matrix factorization technique that enables to impose some prior linear constrains. $LCBMF$ showed good performances in \cite{superbehaviors}, and noting that our task is similar (extracting clusters from smartphone logs), we apply this method to our problem and discuss it in more details in \ref{4.2}. 

