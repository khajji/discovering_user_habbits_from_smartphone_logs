% Chapter 1

\chapter{Introduction} % Main chapter title

\label{Chapter1} % For referencing the chapter elsewhere, use \ref{Chapter1} 

\lhead{Chapter 1 \emph{Introduction}} % This is for the header on each page - perhaps a shortened title

%----------------------------------------------------------------------------------------


Today, data comes from everywhere and in any kind. It streams from daily life; from computers, credit cards, TVs, cars,  sports shoes and watch. The availability of this data is changing the way companies lead their business and are modifying the rules of competitiveness. In a publication that zooms in the challenges and the power of big data, Mc Kinsey affirms that "the use and the understanding of the data will become a key basis of competition and growth for individual firms" and they estimate that a retailer exploiting his data to the full has the potential to increase its operating margin by more than $60$ percent.\par

In this work, we are interested in the data collected from a gadget that shares the life of the user; his smartphone. Smartphone data contains the locations a user visits, the activities he does, the notifications he receives, the applications he launches and many other information. This kind of data is unique in the sense that it represents a complete snapshot of a user's life. In a world driven by the power of data and the ability to anticipate and understand the needs of users, companies shows a big interest in studying this emerging kind of dataset. In the other hand, the richness  and the diversity of this data attracts the curiosity of researchers.\par

In this context, many works have been done with this kind of logs, many paths have been explored and multiple questions answered. Those researches are discussed later with more details.
In this work, we tackle an important question that escaped to the interest of previous researches: Having the smartphone logs of one user, can we find a model that exhibits his particular behaviors and habits? More practically, let's imagine the following example. Let's imagine that Bob has some particular habits: he does running while listening music on Saturdays, he visits his parents on Sundays, he reads news when he is in his office in the mornings, and he puts an alarm clock at 7am during his working days. The question we are answering in this work is the following: Can we find a model that discovers those particular behaviors by analyzing Bob smartphone's logs? How precise can this model be in doing this task?
It is important to note that we are interested in discovering the individual behaviors of a user. Moreover, we choose to build a model that preserves the privacy of the user. For this reason, we aim to find a method that takes as input the logs of a unique user so that it can be applied to a data that never exits his phone. \par

In the recent few years, key innovations allowed smartphones to drastically evolve from cell-phone devices used for calling to powerful "pocket-computers" devices that can be used as cell-phone, camera, calendar, clock, game consol, web browser and many other roles at the same time. As an important company in the smartphone industry, Sony is one of the actors of the smartphones evolution and is aiming on keeping innovating this sector. 
\\Our work is a part of this continuous research for innovation. Indeed, it aims to allow smartphones building a personal relationship with their owner by adapting to their specific needs and answering their specific requests. Let's keep the parallel with Bob to understand what does building a personal relationship with a user concretely means. Let's suppose that Bob's smartphone is able to learn the specific habits of Bob. When Bob forgets to put his alarm clock on a working day, his smartphone can remind him to do it. When an unusual traffic congestion appears in Sunday in the rode that Bob use to take to reach his parents place, his smartphone can inform him. Finally, when his smartphone does not have enough power to play music on Saturday morning, Bob's smartphone can remind him to recharge it because he will probably need it for running. Our work is a start in making it possible for a smartphone to adapt to it's owner's behaviors and to react interactively in some contexts. In other words, it is a start in making smartphones behaves smarter. \par

Scientifically speaking, this problem can be seen as a clustering problem: our goal is to find different clusters of data points where each cluster represents a particular behavior of the user. Coming back to Bob, running while listening to music on Saturdays morning can be represented by a cluster that contains the running activity, the application launch music, the day Saturday and the time frame 8am-12am. Putting an alarm clock at 7 pm during the working days can be represented as a cluster containing all the days of the week, the notification alarm and the time frame 7pm-8am. 
\\Clustering is a widely addressed problem in machine learning and data analysis, and it has been applied to many contexts and topics. It as been used for example in corpus-text modeling, recommender systems and image recognition. Different approaches have been developed to answer those challenges; probabilistic latent topic modeling and matrix factorization are examples of these different approaches. A parallel between our methods and these approaches is made multiple times in this thesis and it will be shown that it is of a strong benefit. 
\\Our problem sits at the interface between an emerging area of research that takes profit of the existence of a new kind of data and an area that constitutes one of the basis of the emergence of the machine learning and data analysis techniques. From a scientifically point fo view, addressing the clustering problem in a new emerging context makes our problem particularly challenging. \par

The thesis is organized as follows: In chapter 2, we introduce some notations and definitions, state the problem in a mathematical way and go through the researches done in this field. 
\\In chapter 3, we describe in details the $Dirichlet$ $Latent$ $Multimodal$ $Representation$ ($DLMR$). It is model that we developed specifically to answer our needs and that shows to perform better than the other existing methods.
\\In chapter 4, we introduce other known and widely used models that has been shown to perform very well in doing tasks similar to ours. We use those models as baselines to evaluate the performances of $DLMR$. To have a complete overview, we both use some models based on the matrix factorization approach using some sophisticated techniques and others based on advanced methods of probabilistic latent topic modeling.
\\In chapter 5, we detail the metrics used to test the performance of the different models in performing the task needed.
\\In chapter 6, we present the results obtained with the different models and compare the performances of the different models to $DLMR$.
\\Finally, chapter 7 presents our conclusions.

%----------------------------------------------------------------------------------------